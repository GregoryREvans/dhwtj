\documentclass[11pt]{article}
\usepackage{fontspec}
\usepackage[utf8]{inputenc}
\setmainfont{Bell MT}
\usepackage[paperwidth=9in,paperheight=12in,margin=1in,headheight=0.0in,footskip=0.5in,includehead,includefoot,portrait]{geometry}
\usepackage[absolute]{textpos}
\TPGrid[0.5in, 0.25in]{23}{24}
\parindent=0pt
\parskip=12pt
\usepackage{nopageno}
\usepackage{graphicx}
\graphicspath{ {./images/} }
\usepackage{amsmath}
\usepackage{tikz}
\newcommand*\circled[1]{\tikz[baseline=(char.base)]{
            \node[shape=circle,draw,inner sep=1pt] (char) {#1};}}

\begin{document}

\begin{center}
\huge INSTRUMENTATION
\end{center}

All performers use the same instrumental setup and therefore the following should be reproduced four times:
\\
\\
\hspace*{1cm} Instruments:
\\
\hspace*{2cm} Bass Drum [x1]
\\
\hspace*{2cm} Tom-toms [x2]
\\
\hspace*{2cm} Congas [x2]
\\
\hspace*{2cm} Bongos [x2]
\\
\hspace*{2cm} Extra Music Stand (empty for performance)
\\
\hspace*{1cm} Implements:
\\
\hspace*{2cm} Mallets suitable for all drums
\\
\hspace*{2cm} Bow

\begin{center}
\huge PERFORMANCE NOTES
\end{center}
\begingroup
\begin{center}

\leftskip0.25in
\pmb{Drums} : Instruments are typically notated in sub-groups to reduce notational complexity, however in passages where the entire instrumental setup is required, a textual indication "drums" will be given. These passages are notated on a six-line staff. The instruments are represented by the spaces within the staff including the spaces below and above it. From low to high the spaces represent the bass drum, low tom-tom, high tom-tom, low-conga, high-conga, low-bongo, and high-bongo.
\rightskip\leftskip
\phantom{text} \hfill \phantom{()}

\leftskip0.25in
\pmb{Bowing} : In passages where the performers are meant to bow a music stand, there is notation which conveys both where the bow is to be placed and a location where the vibrations are to be damped with a light touch from a single finger. A square represents the music stand where a dot represents the damping location and a tick-mark represents the bowing location. When a dotted line with an arrow connects two diagrams, the positions indicated by the first diagram should gradually change to the positions indicated in the second diagram.
\rightskip\leftskip
\phantom{text} \hfill \phantom{()}

\leftskip0.25in
\pmb{Scraping} : Occasionally the music stand is to be scraped. This should be done with the shaft of one of the mallets on one of the edges of the music stand. While scraping, the performer may freely slide the mallet lengthwise to produce higher and lower pitch contours.
\rightskip\leftskip
\phantom{text} \hfill \phantom{()}

\leftskip0.25in
\pmb{Rolls} : Rolls should be performed as fast as possible and not as a measured subdivision of the duration to which they are attached.
\rightskip\leftskip
\phantom{text} \hfill \phantom{()}

\end{center}
\endgroup

\vspace*{9\baselineskip}


\vspace*{26\baselineskip}

\begin{center}
duration: c. 7'
\end{center}

\end{document}
