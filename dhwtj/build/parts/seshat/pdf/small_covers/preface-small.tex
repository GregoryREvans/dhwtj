\documentclass[11pt]{article}
\usepackage{fontspec}
\usepackage[utf8]{inputenc}
\setmainfont{Bell MT}
\usepackage[paperwidth=9.5in,paperheight=11in,margin=1in,headheight=0.0in,footskip=0.5in,includehead,includefoot,portrait]{geometry}
\usepackage[absolute]{textpos}
\TPGrid[0.5in, 0.25in]{23}{24}
\parindent=0pt
\parskip=12pt
\usepackage{nopageno}
\usepackage{graphicx}
\graphicspath{ {./images/} }
\usepackage{amsmath}
\usepackage{tikz}
\newcommand*\circled[1]{\tikz[baseline=(char.base)]{
            \node[shape=circle,draw,inner sep=1pt] (char) {#1};}}

\begin{document}

\begin{center}
\huge INFORMAL INTRODUCTORY LETTER
\end{center}

Hello. I wanted to say a few words before you look at this score. First, this is the first time I have attempted to compose for the voice! I'm certain there are many things which can be improved and I look forward to workshopping this piece and learning from the experience. I hope we can make revisions by hand throughout the collaboration. Thanks for your patience. Also due to a software bug in my notation program I was not able to typeset as many dynamic indications as I wanted, hopefully we can clear up the intentions together and feel free to be creative in the meantime.

Second, this piece is part of a large-scale theatrical work which is currently in progress. If knowledge of the rest of this project would be helpful or interesting, feel free to email me at gregoryrowlandevans@gmail.com to chat about it.

Third, because several languages are used in the text, all sounds are represented in the score in IPA. I am not incredibly familiar with the system so I tried my best. The following pages featuring the text of the piece are written in the original languages, my IPA transliteration, and a translation. The texts are drawn from \textit{Invisible Cities} by Italo Calvino and a variety of poems by Pablo Neruda.

\begin{center}
\huge PERFORMANCE NOTES
\end{center}
\begingroup
\begin{center}

\leftskip1.25in
\pmb{Staff Notation} : Occasionally, the soprano line is written in graphic notation on a three-line staff. The three-line staff suggests the three registers of the speaking voice: low, middle, and high. A ledger line above the staff indicates the highest possible register and a line below the staff indicates the lowest possible register. The curved lines suggest the contour of the utterance of the text. The performer may freely interpret these graphics as the sound they imagine. As a reference, the sound I have in mind is closer to speaking rather than singing. However, occasional breaks of singing or half-singing are welcomed. Pitches indicated on the usual five- line staff are always sung. The soprano lines that are written on a single-line staff are spoken words.
\rightskip\leftskip
\phantom{text} \hfill \phantom{()}

\leftskip1.25in
\pmb{Note Heads} : Some special techniques are represented by the shape of the note head. Slash notation indicates ``perforated'' or ``hoarse'' vocalization, possibly even the use of vocal fry. ``x'' notation indicates that the sound should be ``airy'' while retaining pitch in the noise. Sprechstimme is notated with a cross on the stem.
\rightskip\leftskip
\phantom{text} \hfill \phantom{()}

\leftskip1.25in
\pmb{Miscellaneous} : Inhales and exhales are notated with upbow and downbow symbols respectively. Specialized vibrato contours are indicated with a wavy line above the staff.
\rightskip\leftskip
\phantom{text} \hfill \phantom{()}


\end{center}
\endgroup

\vspace*{9\baselineskip}


\vspace*{26\baselineskip}

\begin{center}
duration: c. 5'
\end{center}

\end{document}
